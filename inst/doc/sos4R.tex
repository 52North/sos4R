%
% http://cran.r-project.org/doc/manuals/R-exts.html#Writing-package-vignettes
% (Code chunks with option eval=FALSE are not tested.)
%

\documentclass{article}

%\VignetteIndexEntry{sos4R: Accessing Sensor Observation Services from R}

\usepackage{graphicx}
\usepackage[colorlinks=true,urlcolor=blue]{hyperref}

\usepackage{color}

\usepackage{Sweave}
\newcommand{\strong}[1]{{\normalfont\fontseries{b}\selectfont #1}}
\let\pkg=\strong

\begin{document}

\title{Accessing Data from Sensor Observation Services:\\ the {\tt sos4R} Package}
\author{Daniel N\"{u}st\footnote{Institute for Geoinformatics, University of M\"{u}nster, Germany.}\\ {\tt daniel.nuest@uni-muenster.de}}

\date{Sep 2010}

\maketitle
\tableofcontents

\section{Introduction}

The {\tt sos4R} package provides classes and methods for dealing with
%spatial data in S (R and S-Plus\footnote{our primary efforts target R;
%depending on the needs, we will address S-Plus as well}). The spatial
%data structures implemented include points, lines, polygons and grids;
%each of them with or without attribute data.  We have chosen to use S4
%classes and methods style (Chambers, 1998) to allow validation of objects
%created. Although we mainly aim at using spatial data in the geographical
%(two-dimensional) domain, the data structures that have a straightforward
%implementation in higher dimensions (points, grids) do allow this.
%
%The motivation to write this package was born on a
%\href{http://spatial.nhh.no/meetings/vienna/index.html}{pre-conference
%spatial data workshop} during
%\href{http://www.ci.tuwien.ac.at/Conferences/DSC-2003/}{DSC 2003}.
%At that time, the advantage of having multiple R packages for spatial
%statistics seemed to be hindered by a lack of a uniform interface for
%handling spatial data. Each package had its own conventions on how
%spatial data were stored and returned. With this package, and packages
%supporting the classes provided here, we hope that R will become a more
%coherent tool for analyzing different types of spatial data.
%
%The package is available, or will be available soon on CRAN. From the
%package home page, \url{http://r-spatial.sourceforge.net/}, a graph
%gallery with R code, and the development source tree are available.
%
%This vignette describes the classes, methods and functions provided
%by sp. Instead of manipulating the class slots (components) directly,
%we provide methods and functions to create the classes from elementary
%types such as matrices, data.frames or lists and to convert them back
%to any of these types. Also, coercion (type casting) from one class to
%the other is provided, where relevant.

Package {\tt sos4R} is loaded by 


\subsection{OGC and Specifications}

TODO: Shortly mention and link to O and M, OWS Common, Sampling, Filtering, SensorML ...

TODO: Round-up of SOS specification

TODO: Copy terms example from annex B of O and M to the vignette for explanation

\section{SOS Requests}

\subsection{GetCapabilities}


\subsection{DescribeSensor}


\subsection{GetObservation}


\subsection{GetObservationById}

\section{Creating a SOS connection}

%\begin{itemize}
%\item \verb|[| select "rows" (items) and/or columns in the data attribute
%table; e.g. {\tt meuse[1:2, "zinc"]} returns a {\tt SpatialPointsDataFrame}
%with the first two points and an attribute table with only variable "zinc".
%\item \verb|[[| select a column from the data attribute table
%\item \verb|[[<-| assign or replace values to a column in the data attribute
%table.
%\end{itemize}


\section{Requesting Data}

\subsection{Using Metadata}

How can one extract the metadata from a SOS connection and reuse it for queries?

accessor functions, elements of the capabilities, ...


\subsection{Making the Request}

\subsection{Data Format}

explain why the returned data.frame is how it looks like

how to extract metadata


\subsection{Using the Data}

show/explain conversion to zoo, sp etc.

exporting

show example

\section{Changing Handling Functions}

TODO: explain approach, mention available non-exchangeable functions in the subsections

\subsection{Parsing/Decoding}

\subsection{Encoding}



\section{Exception Handling}

TODO: Explain what part of the exception report means what, link to OWS Common


\section{Developing sos4R}

explain 52 North

svn code repository

shortly describe the class structure, .R file naming scheme, ...

link to wiki developer page

link to developer blog

\section*{References}
\begin{description}
\item Chambers, J.M., 2008, Software for Data Analysis, Programming with R.
Springer, New York.
\item specs!
\item ...
\end{description}

\end{document}

