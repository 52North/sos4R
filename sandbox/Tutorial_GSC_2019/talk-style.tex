% Common settings for all lectures in this course
\usepackage[utf8]{inputenc}

\usepackage{amsfonts}
\usepackage{amsmath}
\usepackage{amsthm}
\usepackage{amssymb}
\usepackage{bm}
\usepackage{bbm}
% \usepackage[ngerman]{babel}
\usepackage{url}
\usepackage{colortbl}
\usepackage{animate}
\usepackage{pgfpages}
% \pgfpagesuselayout{2 on 1}[a4paper,border shrink=5mm]

% mathSymbols, partly OOo like
\newcommand{\setN}{\mathbb{N}}
\newcommand{\setR}{\mathbb{R}}
\newcommand{\ii}{{\rm i}}

\newcommand{\ui}{{\bf I}}
\newcommand{\us}{{\bf I}^2}
\newcommand{\pmP}{\mathbbm{P}} 		% probability measure IP
\newcommand{\cIR}{\overline{\mathbbm{R}}} 	% closure of IR
\newcommand{\IR}{\mathbbm{R}}			% set of IR
\newcommand{\IE}{\mathbbm{E}}			% IE
\newcommand{\IP}{\mathbbm{P}} 			% probability measure IP
\newcommand{\IN}{\mathbbm{N}} 			% set of IN
\newcommand{\iI}{ \mathcal{I}}
\newcommand{\sU}{\mathrm{\bf U}} 		% sample U
\newcommand{\sV}{\mathrm{\bf V}} 		% sample V
\newcommand{\sX}{\mathrm{\bf X}} 		% sample X
\newcommand{\sY}{\mathrm{\bf Y}} 		% sample Y
\newcommand{\sZ}{\mathrm{\bf Z}} 		% sample Z
\newcommand{\bK}{\mathrm{\bf K}} 		% bold K
\newcommand{\sXY}{\mathrm{\bf X,Y}} 		% sample X,Y
\newcommand{\dom}{\mathrm{dom}} 		% domain
\newcommand{\ran}{\mathrm{ran}} 			% range
\newcommand{\rank}{\mathrm{rank}} 		% rank
\newcommand{\mug}{\text{\textmu g}}		% upright µg used in mathmode or \unitfrac
\newcommand{\mum}{\text{\textmu m}}		% upright µm used in mathmode or \unitfrac


\newcommand{\ST}{\IR^2 \times \IR}		% Space and Time
\newcommand{\R}{ \verb?R? }			% verbatim R

\newcommand{\Note}{\alert{\bf Note:\\}}
\newcommand{\Atte}{\alert{\bf Attention:\\}}

\def\lecturename{Geospatial Sensing Conference 2019} %conference title, section title

\title{\insertlecture}

\author{Eike H. Jürrens,\linebreak Benedikt Gräler,\linebreak Daniel Nüst}

\institute
{
52°North GmbH\\
{\small\url{https://52north.org}}
}

\subject{\lecturename}

%\AtBeginSubsection[]
%{
%  \begin{frame}<beamer>{Outline}
%    \tableofcontents[currentsection,currentsubsection]
%  \end{frame}
%}

\definecolor{blue52n}{HTML}{65c6e4}
\definecolor{grey52n}{HTML}{4f504a}

% If you wish to uncover everything in a step-wise fashion, uncomment
% the following command: 
% \beamerdefaultoverlayspecification{<+->}



% Beamer version theme settings

\useoutertheme[height=0pt,width=2cm,right]{sidebar}
\usecolortheme{rose,sidebartab}
\useinnertheme{rectangles}
\usefonttheme[only large]{structurebold}

\setbeamercolor{sidebar right}{bg=black!10, fg=grey52n}
\setbeamercolor{structure}{fg=blue52n}
\setbeamercolor{author}{parent=structure}
\setbeamercolor{alerted text}{fg=blue52n}
\setbeamercolor{titlelike}{fg=grey52n,bg=black!10}
\setbeamercolor{section in sidebar}{fg=grey52n}

\setbeamerfont{title}{series=\normalfont,size=\LARGE}
\setbeamerfont{title in sidebar}{series=\large}
\setbeamerfont{author in sidebar}{series=\normalfont}
\setbeamerfont*{item}{series=}
\setbeamerfont{frametitle}{size=}
\setbeamerfont{block title}{size=\small}
\setbeamerfont{subtitle}{size=\normalsize,series=\normalfont}
\setbeamerfont{alerted text}{shape=\itshape}

%\rowcolors{2}{white}{black!10}
\hypersetup{colorlinks,linkcolor=,urlcolor=blue52n}

\setbeamertemplate{navigation symbols}{}
\setbeamertemplate{bibliography item}[book]
\setbeamertemplate{sidebar right}
{
  {\usebeamerfont{title in sidebar}%
    \vskip1.5em%
    \hskip3pt%
    \usebeamercolor[fg]{title in sidebar}%
    \insertshorttitle[width=1.8cm,center,respectlinebreaks]\par%
    \vskip1.25em%
  }%
  {%
    \hskip1pt%
    \usebeamercolor[fg]{author in sidebar}%
    \usebeamerfont{author in sidebar}%
    \insertshortauthor[width=2cm,center,respectlinebreaks]\par%
    \vskip1.25em%
  }%
  \hbox to2cm{\hss\insertlogo\hss}
  \vskip1.25em%
  \insertverticalnavigation{2cm}%
  \vfill
  \hbox to 2cm{\hfill\usebeamerfont{subsection in
      sidebar}\strut\usebeamercolor[fg]{subsection in
      sidebar}\insertframenumber\hskip5pt}%
  \vskip3pt%
}%

\setbeamertemplate{title page}
{
  \vbox{}
  \vskip1em
%  {\huge Chapter \insertshortlecture\par}
  {\usebeamercolor[fg]{title}\usebeamerfont{title}\inserttitle\par}%
 \ifx\insertsubtitle\@empty%
 \else%
    \vskip0.25em%
    {\usebeamerfont{subtitle}\usebeamercolor[fg]{subtitle}\insertsubtitle\par}%
 \fi%     
  \vskip1em\par
  \emph{\lecturename} \\ 
  \insertdate\par
  \vskip0pt plus1filll
  \leftskip=0pt plus1fill\insertauthor\par
  \insertinstitute\vskip1em
}

\logo{\includegraphics[width=1.8cm]{Logo_52North.png}}


% Typesetting Listings

\usepackage{listings}
\lstset{language=R}

\alt<presentation>
{\lstset{%
  basicstyle=\footnotesize\ttfamily,
  commentstyle=\slshape\color{green!50!black},
  keywordstyle=\bfseries\color{red!50!black},
  identifierstyle=\color{blue},
  stringstyle=\color{orange},
  escapechar=\#,
  emphstyle=\color{red}}
}
{
  \lstset{%
    basicstyle=\ttfamily,
    keywordstyle=\bfseries,
    commentstyle=\itshape,
    escapechar=\#,
    emphstyle=\bfseries\color{red}
  }
}



% Common theorem-like environments

\theoremstyle{definition}
\newtheorem{exercise}[theorem]{\translate{Exercise}}




% New useful definitions:

\newbox\mytempbox
\newdimen\mytempdimen

\newcommand\includegraphicscopyright[3][]{%
  \leavevmode\vbox{\vskip3pt\raggedright\setbox\mytempbox=\hbox{\includegraphics[#1]{#2}}%
    \mytempdimen=\wd\mytempbox\box\mytempbox\par\vskip1pt%
    \fontsize{3}{3.5}\selectfont{\color{black!10}{\vbox{\hsize=\mytempdimen#3}}}\vskip3pt%
}}

\newenvironment{colortabular}[1]{\medskip\rowcolors[]{1}{blue!20}{blue!10}\tabular{#1}\rowcolor{blue!40}}{\endtabular\medskip}

\def\equad{\leavevmode\hbox{}\quad}

\newenvironment{greencolortabular}[1]
{\medskip\rowcolors[]{1}{green!50!black!20}{green!50!black!10}%
  \tabular{#1}\rowcolor{green!50!black!40}}%
{\endtabular\medskip}



